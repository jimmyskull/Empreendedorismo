\chapter{Estrutura e operação} 

A estratégia de operação está preocupada menos com os processos individuais
e mais com o processo de transformação total do negócio como um todo.  Está
centrado em como o ambiente competitivo está mudando e o que a operação deve
fazer para esteja preparada para atender desafios atuais e futuros
\cite{nigel2009operations}.  Também está preocupada com o desenvolvimento
a longo prazo dos recursos de suas operações e dos processos para que possam
ser utilizados como uma vantagem sustentável.

A organização estrutural de uma empresa deve estar bem definida. Deve-se 
apresentar pelo menos um organograma da empresa e/ou da equipe, além de um 
cronograma dos processos operativos que serão executados. Juntamente com 
estes dados gráficos, é interessante uma descrição breve de cada um, para 
fins explicativos \cite{slack2002administração}. Não obstante, é interessante 
uma tabela com custos administrativos (fixos) e operacionais (variáveis).

